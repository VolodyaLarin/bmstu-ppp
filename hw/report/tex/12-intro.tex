\Introduction


Целью данной работы является разработка программы для вычисления матрицы, обратной заданной, с использованием метода Р-приведения.

Для достижения цели необходимо выполнить следующие задачи:
\begin{itemize}
    \item обосновановать выбор алгоритма решения задачи;
    \item разработать программу для нахождения обратной матрицы методом Р-приведения:
    \item исследование влияния размерности матрицы и числа процессоров на время выполнения программы.
\end{itemize}


Актуальность исследования обусловлена необходимостью разработки эффективных методов для решения задач линейной алгебры, таких как вычисление обратной матрицы, которые могут применяться в различных областях науки и техники.
Обратные матрицы играют ключевую роль во многих математических моделях, например, в системах управления, обработке сигналов, статистическом анализе данных и других приложениях.

Метод Р-приведения является одним из классических подходов к решению подобных задач, однако его эффективность может значительно варьироваться в зависимости от размера задачи и архитектуры вычислительного оборудования.
Современные высокопроизводительные системы требуют оптимизированных решений, способных равномерно распределять нагрузку между несколькими процессорами, что позволяет существенно сократить время расчётов.

Исследование зависимости времени счёта от размерности задачи и количества процессоров позволит выявить закономерности и определить наиболее эффективные параметры для реализации программ на современных многоядерных системах.
Это знание будет полезно не только для улучшения производительности существующих приложений, но и для проектирования новых систем, ориентированных на параллельную обработку данных.


