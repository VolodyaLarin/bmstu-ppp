\chapter{Экспериментальный раздел}
\label{cha:research}


\section{Технические характеристики}
Вычисления проводились на кластере, состоящем из 10 узлов, со следующими техническими характеристиками:
\begin{itemize}
    \item Операционная система Ubuntu 18.04.6 LTS (GNU/Linux 4.15.0 x86\_64)
    \item Оперативная память 4 ГБ
    \item Процессор Intel(R) Xeon(R) Gold 6248 CPU @ 2.50GHz --- 1 ядро
\end{itemize}

Для управления задачами использовались системы управления заданиями Slurm.


\section{Время выполнения алгоритмов}

Для замеров времени использовалась функция MPI \Verb{MPI_Wtime}.

Для одного узла используется последовательнный алгоритм.

Было вычислено ускорение вычислений $S$ по формуле
\begin{equation}
    \label{eq:eq_s}
    S(Size, n) = \frac{Time(Size, 1)} {Time(Size, n)}
\end{equation}
и коэффициент полезной нагрузки $E$ по формуле
\begin{equation}
    \label{eq:eq_e}
    E(Size, n) = \frac{S(Size, n)} {n} ,
\end{equation}
где $Size$ -- размерность квадратной матрицы, $n$ -- количество вычислительных узлов.

Результаты замеров и вычислений представлены в таблице~\ref{tab:results-all-m}.


\begin{table}[H]
    \center

    \caption{Результаты вычислений}
    \label{tab:results-all-m}
    \fontsize{11pt}{11pt}\selectfont
    \begin{tabular}{|l|l|rrrrrrrrrr|}
        \hline
        & PC   & 1      & 2      & 3      & 4      & 5      & 6      & 7      & 8     & 9     & 10    \\
        Size                     &      &        &        &        &        &        &        &        &       &       &       \\
        \hline
        \multirow[t]{3}{*}{100}  & E    & 1.000  & 0.138  & 0.085  & 0.050  & 0.038  & 0.013  & 0.011  & 0.018 & 0.008 & 0.015 \\
        & S    & 1.000  & 0.276  & 0.256  & 0.202  & 0.190  & 0.079  & 0.076  & 0.142 & 0.073 & 0.150 \\
        & Time & 0.006  & 0.023  & 0.025  & 0.032  & 0.034  & 0.081  & 0.085  & 0.045 & 0.088 & 0.043 \\
        \cline{1-12}
        \multirow[t]{3}{*}{200}  & E    & 1.000  & 0.359  & 0.255  & 0.154  & 0.153  & 0.107  & 0.087  & 0.061 & 0.069 & 0.053 \\
        & S    & 1.000  & 0.718  & 0.764  & 0.615  & 0.764  & 0.643  & 0.608  & 0.484 & 0.619 & 0.526 \\
        & Time & 0.049  & 0.068  & 0.064  & 0.080  & 0.064  & 0.076  & 0.081  & 0.101 & 0.079 & 0.093 \\
        \cline{1-12}
        \multirow[t]{3}{*}{400}  & E    & 1.000  & 0.643  & 0.534  & 0.381  & 0.381  & 0.289  & 0.242  & 0.197 & 0.195 & 0.161 \\
        & S    & 1.000  & 1.285  & 1.601  & 1.525  & 1.903  & 1.736  & 1.697  & 1.573 & 1.757 & 1.609 \\
        & Time & 0.381  & 0.296  & 0.238  & 0.250  & 0.200  & 0.219  & 0.224  & 0.242 & 0.217 & 0.237 \\
        \cline{1-12}
        \multirow[t]{3}{*}{600}  & E    & 1.000  & 0.793  & 0.677  & 0.474  & 0.460  & 0.449  & 0.363  & 0.291 & 0.274 & 0.248 \\
        & S    & 1.000  & 1.587  & 2.032  & 1.895  & 2.302  & 2.696  & 2.542  & 2.325 & 2.469 & 2.475 \\
        & Time & 1.283  & 0.809  & 0.631  & 0.677  & 0.557  & 0.476  & 0.505  & 0.552 & 0.520 & 0.518 \\
        \cline{1-12}
        \multirow[t]{3}{*}{800}  & E    & 1.000  & 0.535  & 0.729  & 0.335  & 0.268  & 0.256  & 0.409  & 0.357 & 0.149 & 0.143 \\
        & S    & 1.000  & 1.069  & 2.186  & 1.340  & 1.341  & 1.537  & 2.862  & 2.854 & 1.345 & 1.434 \\
        & Time & 3.015  & 2.819  & 1.379  & 2.249  & 2.248  & 1.962  & 1.053  & 1.056 & 2.241 & 2.102 \\
        \cline{1-12}
        \multirow[t]{3}{*}{1000} & E    & 1.000  & 0.876  & 0.826  & 0.470  & 0.407  & 0.357  & 0.310  & 0.347 & 0.266 & 0.237 \\
        & S    & 1.000  & 1.753  & 2.479  & 1.881  & 2.037  & 2.141  & 2.171  & 2.775 & 2.390 & 2.375 \\
        & Time & 5.881  & 3.356  & 2.372  & 3.127  & 2.888  & 2.747  & 2.709  & 2.119 & 2.461 & 2.477 \\
        \cline{1-12}
        \multirow[t]{3}{*}{1200} & E    & 1.000  & 0.930  & 0.890  & 0.637  & 0.513  & 0.473  & 0.437  & 0.483 & 0.351 & 0.332 \\
        & S    & 1.000  & 1.861  & 2.671  & 2.547  & 2.565  & 2.835  & 3.057  & 3.867 & 3.155 & 3.319 \\
        & Time & 10.517 & 5.652  & 3.937  & 4.129  & 4.100  & 3.709  & 3.441  & 2.720 & 3.334 & 3.169 \\
        \cline{1-12}
        \multirow[t]{3}{*}{1400} & E    & 1.000  & 0.921  & 0.909  & 0.823  & 0.637  & 0.554  & 0.514  & 0.575 & 0.407 & 0.393 \\
        & S    & 1.000  & 1.843  & 2.727  & 3.290  & 3.183  & 3.324  & 3.595  & 4.596 & 3.667 & 3.931 \\
        & Time & 16.734 & 9.081  & 6.137  & 5.086  & 5.257  & 5.034  & 4.655  & 3.641 & 4.563 & 4.257 \\
        \cline{1-12}
        \multirow[t]{3}{*}{1600} & E    & 1.000  & 0.937  & 0.920  & 0.861  & 0.700  & 0.663  & 0.596  & 0.601 & 0.535 & 0.487 \\
        & S    & 1.000  & 1.873  & 2.760  & 3.444  & 3.502  & 3.981  & 4.174  & 4.810 & 4.814 & 4.873 \\
        & Time & 25.403 & 13.562 & 9.204  & 7.377  & 7.254  & 6.382  & 6.086  & 5.281 & 5.277 & 5.213 \\
        \cline{1-12}
        \multirow[t]{3}{*}{1800} & E    & 1.000  & 0.922  & 0.901  & 0.792  & 0.715  & 0.730  & 0.625  & 0.591 & 0.545 & 0.528 \\
        & S    & 1.000  & 1.844  & 2.704  & 3.170  & 3.576  & 4.379  & 4.378  & 4.726 & 4.905 & 5.279 \\
        & Time & 35.814 & 19.422 & 13.244 & 11.299 & 10.014 & 8.179  & 8.181  & 7.578 & 7.301 & 6.784 \\
        \cline{1-12}
        \multirow[t]{3}{*}{2000} & E    & 1.000  & 0.929  & 0.910  & 0.855  & 0.789  & 0.689  & 0.688  & 0.650 & 0.602 & 0.599 \\
        & S    & 1.000  & 1.857  & 2.729  & 3.419  & 3.946  & 4.133  & 4.813  & 5.202 & 5.417 & 5.988 \\
        & Time & 49.325 & 26.559 & 18.077 & 14.425 & 12.501 & 11.935 & 10.248 & 9.481 & 9.106 & 8.238 \\
        \cline{1-12}
        \hline
    \end{tabular}
\end{table}


На рисунке~\ref{fig:graph} показан график зависимости характеристик метода от размерности задачи для разного количества вычислительных узлов.

\begin{figure}[H]
    \centering
    \includesvg[width=\textwidth]{img/plot_1_1.svg}
    \includesvg[width=\textwidth]{img/plot_1_2.svg}
    \caption{Зависимость характеристик работы метода от размерности задачи для разного количества вычислительных узлов}
    \label{fig:graph}
\end{figure}

Согласно полученным результатам можно сделать вывод о том, что использование более одного вычислительного узла оправдано при вычислениях с матрицами размерностью более 200x200.


Были построены тепловые карты зависимости количества вычислительных узлов от размерности задачи для ускорения вычислений $S$ и коэфициента полезной нагрузки $E$.
Данные карты предствлены на рисунке~\ref{fig:heatmap_s}.

\begin{figure}[H]
    \centering
    \begin{minipage}[h]{0.49\textwidth}
        \center{\includesvg[width=1\textwidth]{img/heatmap_s.svg} \\ a)}
    \end{minipage}
    \hfill
    \begin{minipage}[h]{0.49\textwidth}
        \center{\includesvg[width=1\textwidth]{img/heatmap_e.svg} \\ б)}
    \end{minipage}
    \caption{Тепловая карта зависимости количества вычислительных узлов от размерности задачи для \\ a) ускорения вычислений  $S$  б) коэфициента полезной нагрузки  $E$   }
    \label{fig:heatmap_s}
\end{figure}


Из данных тепловых карт можно сделать вывод о том, что до 40\% времени работы алгоритма уходит на операции распределения данных между вычислительными узлами и процесссов их синхронизации.


\section{Время выполнения алгоритмов c использованием OpenMP}

Результаты замеров и вычислений представлены в таблице~\ref{tab:results-all-m-2}.


\begin{table}[H]
    \center

    \caption{Результаты вычислений c использованием OpenMP}
    \label{tab:results-all-m-2}
    \fontsize{11pt}{11pt}\selectfont
    \begin{tabular}{|l|l|rrrrrrrrrr|}
        \hline
        & PC   & 1      & 2     & 3     & 4     & 5     & 6     & 7     & 8     & 9     & 10    \\
        Size                     &      &        &       &       &       &       &       &       &       &       &       \\
        \hline
        \multirow[t]{3}{*}{100}  & E    & 1.000  & 0.143 & 0.095 & 0.062 & 0.040 & 0.026 & 0.024 & 0.016 & 0.015 & 0.015 \\
        & S    & 1.000  & 0.287 & 0.285 & 0.246 & 0.202 & 0.159 & 0.165 & 0.131 & 0.139 & 0.147 \\
        & Time & 0.006  & 0.022 & 0.022 & 0.026 & 0.032 & 0.040 & 0.039 & 0.049 & 0.046 & 0.044 \\
        \cline{1-12}
        \multirow[t]{3}{*}{200}  & E    & 1.000  & 0.462 & 0.289 & 0.188 & 0.147 & 0.090 & 0.077 & 0.053 & 0.059 & 0.043 \\
        & S    & 1.000  & 0.923 & 0.866 & 0.753 & 0.736 & 0.542 & 0.538 & 0.425 & 0.535 & 0.435 \\
        & Time & 0.049  & 0.053 & 0.057 & 0.065 & 0.067 & 0.090 & 0.091 & 0.115 & 0.092 & 0.113 \\
        \cline{1-12}
        \multirow[t]{3}{*}{400}  & E    & 1.000  & 1.140 & 0.801 & 0.575 & 0.478 & 0.303 & 0.264 & 0.192 & 0.195 & 0.153 \\
        & S    & 1.000  & 2.280 & 2.402 & 2.301 & 2.392 & 1.819 & 1.848 & 1.536 & 1.751 & 1.526 \\
        & Time & 0.381  & 0.167 & 0.159 & 0.166 & 0.159 & 0.209 & 0.206 & 0.248 & 0.218 & 0.250 \\
        \cline{1-12}
        \multirow[t]{3}{*}{600}  & E    & 1.000  & 1.549 & 1.247 & 0.905 & 0.788 & 0.566 & 0.477 & 0.333 & 0.409 & 0.313 \\
        & S    & 1.000  & 3.098 & 3.742 & 3.620 & 3.938 & 3.398 & 3.342 & 2.664 & 3.682 & 3.130 \\
        & Time & 1.283  & 0.414 & 0.343 & 0.354 & 0.326 & 0.378 & 0.384 & 0.482 & 0.349 & 0.410 \\
        \cline{1-12}
        \multirow[t]{3}{*}{800}  & E    & 1.000  & 0.856 & 1.647 & 0.452 & 1.040 & 0.278 & 0.594 & 0.431 & 0.183 & 0.379 \\
        & S    & 1.000  & 1.712 & 4.941 & 1.807 & 5.202 & 1.665 & 4.156 & 3.450 & 1.647 & 3.790 \\
        & Time & 3.015  & 1.761 & 0.610 & 1.668 & 0.580 & 1.810 & 0.725 & 0.874 & 1.831 & 0.796 \\
        \cline{1-12}
        \multirow[t]{3}{*}{1000} & E    & 1.000  & 2.335 & 0.974 & 0.786 & 1.304 & 0.427 & 0.371 & 0.326 & 0.278 & 0.280 \\
        & S    & 1.000  & 4.669 & 2.922 & 3.144 & 6.522 & 2.562 & 2.600 & 2.611 & 2.502 & 2.799 \\
        & Time & 5.881  & 1.260 & 2.012 & 1.871 & 0.902 & 2.296 & 2.262 & 2.252 & 2.350 & 2.101 \\
        \cline{1-12}
        \multirow[t]{3}{*}{1200} & E    & 1.000  & 2.014 & 1.486 & 1.116 & 0.788 & 0.703 & 0.636 & 0.545 & 0.463 & 0.419 \\
        & S    & 1.000  & 4.027 & 4.458 & 4.464 & 3.940 & 4.220 & 4.449 & 4.357 & 4.169 & 4.188 \\
        & Time & 10.517 & 2.612 & 2.359 & 2.356 & 2.669 & 2.493 & 2.364 & 2.414 & 2.523 & 2.511 \\
        \cline{1-12}
        \multirow[t]{3}{*}{1400} & E    & 1.000  & 2.124 & 1.783 & 1.194 & 0.960 & 0.937 & 0.790 & 0.724 & 0.625 & 0.508 \\
        & S    & 1.000  & 4.249 & 5.348 & 4.777 & 4.800 & 5.625 & 5.530 & 5.791 & 5.628 & 5.078 \\
        & Time & 16.734 & 3.939 & 3.129 & 3.503 & 3.486 & 2.975 & 3.026 & 2.890 & 2.973 & 3.295 \\
        \cline{1-12}
        \multirow[t]{3}{*}{1600} & E    & 1.000  & 3.033 & 2.333 & 1.545 & 1.217 & 1.120 & 0.924 & 0.821 & 0.760 & 0.739 \\
        & S    & 1.000  & 6.065 & 7.000 & 6.180 & 6.087 & 6.718 & 6.465 & 6.565 & 6.843 & 7.389 \\
        & Time & 25.403 & 4.188 & 3.629 & 4.111 & 4.173 & 3.781 & 3.930 & 3.869 & 3.712 & 3.438 \\
        \cline{1-12}
        \multirow[t]{3}{*}{1800} & E    & 1.000  & 3.110 & 2.487 & 1.877 & 1.434 & 1.258 & 1.101 & 0.955 & 0.828 & 0.798 \\
        & S    & 1.000  & 6.219 & 7.460 & 7.508 & 7.172 & 7.548 & 7.709 & 7.638 & 7.455 & 7.979 \\
        & Time & 35.814 & 5.759 & 4.801 & 4.770 & 4.994 & 4.745 & 4.646 & 4.689 & 4.804 & 4.488 \\
        \cline{1-12}
        \multirow[t]{3}{*}{2000} & E    & 1.000  & 3.067 & 2.444 & 1.893 & 1.782 & 1.495 & 1.312 & 1.107 & 1.046 & 0.968 \\
        & S    & 1.000  & 6.134 & 7.333 & 7.572 & 8.911 & 8.967 & 9.182 & 8.855 & 9.414 & 9.683 \\
        & Time & 49.325 & 8.041 & 6.727 & 6.514 & 5.535 & 5.501 & 5.372 & 5.571 & 5.239 & 5.094 \\
        \cline{1-12}
        \bottomrule
    \end{tabular}
\end{table}



На рисунке~\ref{fig:graph_2} показан график зависимости характеристик метода от размерности задачи для разного количества вычислительных узлов.

\begin{figure}[H]
    \centering
    \includesvg[width=\textwidth]{img/plot_mp_1.svg}
    \includesvg[width=\textwidth]{img/plot_mp_2.svg}
    \caption{Зависимость характеристик работы метода от размерности задачи для разного количества вычислительных узлов с использованием OpenMP}
    \label{fig:graph_2}
\end{figure}

Согласно полученным результатам можно сделать вывод о том, что использование более одного вычислительного узла оправдано при вычислениях с матрицами размерностью более 200x200.


Были построены тепловые карты зависимости количества вычислительных узлов от размерности задачи для ускорения вычислений $S$ и коэффициента полезной нагрузки $E$  c использованием OpenMP.
Данные карты представлены на рисунке~\ref{fig:heatmap_s2}.

\begin{figure}[H]
    \centering
    \begin{minipage}[h]{0.49\textwidth}
        \center{\includesvg[width=1\textwidth]{img/heatmap_s_2.svg} \\ a)}
    \end{minipage}
    \hfill
    \begin{minipage}[h]{0.49\textwidth}
        \center{\includesvg[width=1\textwidth]{img/heatmap_e_2.svg} \\ б)}
    \end{minipage}
    \caption{Тепловая карта c использованием OpenMP зависимости количества вычислительных узлов от размерности задачи для \\ a) ускорения вычислений  $S$  б) коэфициента полезной нагрузки  $E$   }
    \label{fig:heatmap_s2}
\end{figure}
    


\section*{Вывод}

Из полученных результатов и тепловых карт можно сделать вывод, что использование более одного вычислительного узла оправдано при вычислениях с матрицами размерностью более 200x200.
До 40\% времени работы алгоритма уходит на операции распределения данных между вычислительными узлами и процессов их синхронизации.


