\chapter{Аналитический раздел}


\section{Обратная матрица}

Обратная ма́трица — такая матрица $A^{-1}$, при умножении которой на исходную матрицу $A$ получается единичная матрица $E$:

\begin{equation}
    \label{eq:eq_a1}
    AA^{-1}=A^{-1}A=E.
\end{equation}

Обратную матрицу можно определить как:

\begin{equation}
    \label{eq:eq_a2}
    A^{-1}={\frac {{\mbox{adj}}A}{|A|}},
\end{equation}
где ${\mbox{adj}}A$ — соответствующая присоединённая матрица, $|A|$ — определитель матрицы $A$.

Из этого определения следует критерий обратимости: матрица обратима тогда и только тогда, когда она невырождена, то есть её определитель не равен нулю.
Для неквадратных матриц и вырожденных матриц обратных матриц не существует.


\section{Метод Жордана—Гаусса}


Возьмём две матрицы: саму $A$ и единичную матрицу $E$. Приведём матрицу $A$ к единичной методом Гаусса—Жордана, применяя преобразования по строкам (можно также применять преобразования и по столбцам).
После применения каждой операции к первой матрице применим ту же операцию ко второй.
Когда приведение первой матрицы к единичному виду будет завершено, вторая матрица окажется равной $A^{-1}$.

При использовании метода Гаусса первая матрица будет умножаться слева на одну из элементарных матриц $\Lambda_{i}$ (трансвекцию или диагональную матрицу с единицами на главной диагонали, кроме одной позиции):

\begin{equation}
    \label{eq:eq_a10}
    \Lambda _{1}\cdot \dots \cdot \Lambda _{n}\cdot A=\Lambda A=E\Rightarrow \Lambda =A^{-1}.
\end{equation}

\begin{equation}
    \label{eq:eq_a11}
    \Lambda _{m}={\begin{bmatrix}
                      1& \dots & 0& -a_{1m}/a_{mm}& 0& \dots & 0\\& & & \dots & & & \\0& \dots & 1& -a_{m-1m}/a_{mm}& 0& \dots & 0\\0& \dots & 0& 1/a_{mm}& 0& \dots & 0\\0& \dots & 0& -a_{m+1m}/a_{mm}& 1& \dots & 0\\& & & \dots & & & \\0& \dots & 0& -a_{nm}/a_{mm}& 0& \dots & 1
    \end{bmatrix}}.
\end{equation}

Вторая матрица после применения всех операций станет равна $\Lambda$, то есть будет искомой.



\subsection*{Описание алгоритма}

Алгоритм решения подразделяется на два этапа.


На первом этапе осуществляется прямой ход, в ходе которого происходит следующее:


\begin{enumerate}
    \item Определяется, является ли система уравнений совместной. Для этого среди элементов первого столбца матрицы выбирается ненулевой элемент.
    \item Строка, содержащая этот ненулевой элемент, перемещается в крайнее верхнее положение, становясь первой строкой матрицы.
    \item Ненулевые элементы первого столбца всех нижележащих строк обнуляются путём вычитания из каждой строки первой строки, умноженной на отношение первого элемента этой строки к первому элементу первой строки.
\end{enumerate}

После выполнения указанных преобразований первая строка и первый столбец мысленно вычёркиваются, и процесс повторяется до тех пор, пока не останется матрица нулевого размера.


На втором этапе осуществляется обратный ход, в ходе которого происходит следующее:


\begin{enumerate}
    \item Выражается каждая базисная переменная через небазисные переменные.
    \item Строится фундаментальная система решений.
    \item Если все переменные являются базисными, то выражается единственное решение системы линейных уравнений.
\end{enumerate}

Процедура начинается с последнего уравнения, из которого выражается соответствующая базисная переменная (она там всего одна) и подставляется в предыдущие уравнения.
Затем процесс повторяется для каждого уравнения, поднимаясь по «ступеням» наверх.
Каждой строке соответствует ровно одна базисная переменная, поэтому на каждом шаге, кроме последнего, ситуация повторяется.

Сложность алгоритма -- $O(n^{3})$.


