\chapter{Технологический раздел}
\label{cha:impl}


\section{Выбор средств программной реализации}


Основным средством разработки является язык программирования.
Был выбран язык программирования C.
Для выполенения паралелльныхх вычислений была использована библиотека OpenMPI.


\section{Структуры MPI}


Был создан тип данных, представляющий собой непрерывную последовательность элементов заданного базового типа MPI\_DOUBLE.
Этот тип данных удобен для передачи целых строк матрицы.
Данный тип определен следующим образом.

\begin{small}
    \begin{verbatim}
MPI_Datatype mpi_row;
MPI_Type_contiguous(matrix_length, MPI_DOUBLE, &mpi_row);
    \end{verbatim}
\end{small}

Для цикличной рассылки был использован следующий тип.
Эта структура данных описывает создание типа данных для передачи строк матрицы в MPI.
Сначала создается временный тип данных -- вектор из строк матрицы c пробелом в $size-1$ строк.
Затем данный временный тип преобразуется в окончательный тип: задается выравнивание в памяти, чтобы можно было получить следующие $N$ строк для другого узла.
Тип определен следующим образом.


\begin{small}
    \begin{verbatim}
MPI_Datatype mpi_row_sh;
{
    MPI_Datatype mpi_row_tmp;
    MPI_Type_vector(max_matrix_rows, matrix_length,
        matrix_length * size, MPI_DOUBLE, &mpi_row_tmp);
    MPI_Type_create_resized(mpi_row_tmp, 0,
        sizeof(double) * matrix_length, &mpi_row_sh);
    MPI_Type_free(&mpi_row_tmp);
}
    \end{verbatim}
\end{small}


Отправка и сбор данных происходит с помощью функций.


\begin{small}
    \begin{verbatim}
MPI_Scatter(matrix, 1, mpi_row_sh,
            matrix_r, max_matrix_rows, mpi_row,
            0, MPI_COMM_WORLD);

MPI_Gather(inv_matrix, max_matrix_rows, mpi_row,
           invert_matrix, 1, mpi_row_sh,
           0, MPI_COMM_WORLD);
    \end{verbatim}
\end{small}


\section{Организация проверки валидности данных}


Для проверки ошибки вычислений используется следующая функция.


\begin{small}
    \begin{verbatim}
double inv_check(int matrix_len, double *left, double *right) {
    double *tmp = malloc(sizeof(*tmp) * matrix_len * matrix_len);

    mul_matrix(left, right, tmp, matrix_len);

    double diff = 0;
    for (size_t row = 0; row < matrix_len; row++) {
        for (size_t col = 0; col < matrix_len; col++) {
            double data = tmp[row * matrix_len + col];
            if (row == col) diff += fabs(1 - data);
            else diff += fabs(data);
        }
    }

    free(tmp);
    return diff / (matrix_len * matrix_len);
}
    \end{verbatim}
\end{small}


\section*{Вывод}

В данном разделе описывается выбор средств программной реализации для выполнения определённых задач.
Для разработки используется язык программирования C. Для параллельных вычислений применяется OpenMPI.
Для рассылки строк и цкиличной рассылки строк матрицы были разработаны типы данных.

Для контроля ошибок вычислений применяется следующая функция.
Функция inv\_check умножает матрицы, вычисляет разницу между резудьтатом и единичной матрицей.

