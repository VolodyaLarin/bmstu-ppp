%% Преамбула TeX-файла

% 1. Стиль и язык
\documentclass[utf8x, times, 12pt]{G7-32} % Стиль (по умолчанию размер шрифта 14pt)


% Остальные стандартные настройки убраны в preamble.inc.tex.
\sloppy

% Настройки стиля ГОСТ 7-32
% Для начала определяем, хотим мы или нет, чтобы рисунки и таблицы нумеровались в пределах раздела, или нам нужна сквозная нумерация.
\EqInChapter % формулы будут нумероваться в пределах раздела
\TableInChapter % таблицы будут нумероваться в пределах раздела
\PicInChapter % рисунки будут нумероваться в пределах раздела

% Добавляем гипертекстовое оглавление в PDF
\usepackage[
bookmarks=true, colorlinks=true, unicode=true,
urlcolor=black,linkcolor=black, anchorcolor=black,
citecolor=black, menucolor=black, filecolor=black,
]{hyperref}

\AfterHyperrefFix

\usepackage{microtype}% полезный пакет для микротипографии, увы под xelatex мало чего умеет, но под pdflatex хорошо улучшает читаемость

% Тире могут быть невидимы в Adobe Reader
\ifInvisibleDashes
\MakeDashesBold
\fi

\usepackage{graphicx}   % Пакет для включения рисунков

% С такими оно полями оно работает по-умолчанию:
% \RequirePackage[left=20mm,right=10mm,top=20mm,bottom=20mm,headsep=0pt,includefoot]{geometry}
% Если вас тошнит от поля в 10мм --- увеличивайте до 20-ти, ну и про переплёт не забывайте:
\geometry{right=20mm}
\geometry{left=30mm}
\geometry{bottom=20mm}
\geometry{ignorefoot}% считать от нижней границы текста


% Пакет Tikz
\usepackage{tikz}
\usetikzlibrary{arrows,positioning,shadows}

% Произвольная нумерация списков.
\usepackage{enumerate}

% ячейки в несколько строчек
\usepackage{multirow}

% itemize внутри tabular
\usepackage{paralist,array}

%\setlength{\parskip}{1ex plus0.5ex minus0.5ex} % разрыв между абзацами
\setlength{\parskip}{1ex} % разрыв между абзацами
\usepackage{blindtext}

% Центрирование подписей к плавающим окружениям
%\usepackage[justification=centering]{caption}

\usepackage{newfloat}
\DeclareFloatingEnvironment[
placement={!ht},
name=Equation
]{eqndescNoIndent}
\edef\fixEqndesc{\noexpand\setlength{\noexpand\parindent}{\the\parindent}\noexpand\setlength{\noexpand\parskip}{\the\parskip}}
\newenvironment{eqndesc}[1][!ht]{%
    \begin{eqndescNoIndent}[#1]%
\fixEqndesc%
}
{\end{eqndescNoIndent}}


\usepackage{ulem}
% Дополнительное окружения для подписей
\usepackage{array}

\usepackage{svg}

\svgsetup{inkscapelatex=false}

\usepackage{pdflscape}



\usepackage{algorithm}
\usepackage{algpseudocode}
\floatname{algorithm}{Алгоритм}


\algrenewcommand\algorithmicwhile{\textbf{Пока}}
\algrenewcommand\algorithmicdo{\textbf{выполнить}}
\algrenewcommand\algorithmicrepeat{\textbf{Повторять}}
\algrenewcommand\algorithmicuntil{\textbf{Пока выполняется}}
\algrenewcommand\algorithmicend{\textbf{Конец}}
\algrenewcommand\algorithmicif{\textbf{Если}}
\algrenewcommand\algorithmicelse{\textbf{иначе}}
\algrenewcommand\algorithmicthen{\textbf{тогда}}
\algrenewcommand\algorithmicfor{\textbf{Цикл}}
\algrenewcommand\algorithmicforall{\textbf{Для всех}}
\algrenewcommand\algorithmicfunction{\textbf{Функция}}
\algrenewcommand\algorithmicprocedure{\textbf{Процедура}}
\algrenewcommand\algorithmicloop{\textbf{Зациклить}}
\algrenewcommand\algorithmicrequire{\textbf{Условия:}}
\algrenewcommand\algorithmicensure{\textbf{Обеспечивающие условия:}}
\algrenewcommand\algorithmicreturn{\textbf{Вернуть}}
\algrenewtext{EndWhile}{\textbf{Конец цикла}}
\algrenewtext{EndLoop}{\textbf{Конец зацикливания}}
\algrenewtext{EndFor}{\textbf{Конец цикла}}
\algrenewtext{EndFunction}{\textbf{Конец функции}}
\algrenewtext{EndProcedure}{\textbf{Конец процедуры}}
\algrenewtext{EndIf}{\textbf{Конец условия}}
\algrenewtext{EndFor}{\textbf{Конец цикла}}
\algrenewtext{BeginAlgorithm}{\textbf{Начало алгоритма}}
\algrenewtext{EndAlgorithm}{\textbf{Конец алгоритма}}
\algrenewtext{BeginBlock}{\textbf{Начало блока. }}
\algrenewtext{EndBlock}{\textbf{Конец блока}}
\algrenewtext{ElsIf}{\textbf{иначе если }}

\renewcommand{\thealgorithm}{\thechapter.\arabic{algorithm}}%

\makeatletter
\@addtoreset{algorithm}{chapter}% algorithm counter resets every chapter
\makeatother


% Fix breaking algos per page
\makeatletter
\newenvironment{breakablealgorithm}
{% \begin{breakablealgorithm}
    \begin{center}
        \refstepcounter{algorithm}% New algorithm
%        \hrule height.8pt depth0pt \kern2pt% \@fs@pre for \@fs@ruled
        \renewcommand{\caption}[2][\relax]{% Make a new \caption
                {\raggedright\fname@algorithm~\thealgorithm~---~ ##2\par  }%
            \ifx\relax##1\relax % #1 is \relax
            \addcontentsline{loa}{algorithm}{\protect\numberline{\thealgorithm}##2}%
            \else % #1 is not \relax
            \addcontentsline{loa}{algorithm}{\protect\numberline{\thealgorithm}##1}%
            \fi

%            \kern2pt\hrule\kern2pt
        }
        }{% \end{breakablealgorithm}
%        \kern2pt\hrule\relax% \@fs@post for \@fs@ruled
    \end{center}
}
\makeatother


%\usepackage[style=gost-numeric, % стиль цитирования и библиографии
%    language=auto, % получение языка из babel
%    autolang=other, % многоязычная библиография
%    sorting=unsrt
%%    ...
%]{biblatex}
%\addbibresource{rpz.bib}


\usepackage{tocloft}

\setlength{\cftbeforetoctitleskip}{0pt}
\renewcommand{\cfttoctitlefont}{\hfill\normalsize\bfseries}
\renewcommand{\cftaftertoctitle}{\hfill\mbox{}}


% Remove padding for TOC entries
\renewcommand{\cftsecindent}{0em}
\renewcommand{\cftsubsecindent}{0em}
\renewcommand{\cftsubsubsecindent}{0em}

% Customizing TOC entries font and alignment
\renewcommand{\cftsecfont}{\normalsize}
\renewcommand{\cftsubsecfont}{\normalsize}
\renewcommand{\cftsubsubsecfont}{\normalsize}

\setlength{\cftbeforetoctitleskip}{-20pt}


% Настройки листингов.
\ifPDFTeX
% 8 Листинги

\usepackage{listings}

% Значения по умолчанию
\lstset{
  basicstyle= \footnotesize,
  breakatwhitespace=true,% разрыв строк только на whitespacce
  breaklines=true,       % переносить длинные строки
%   captionpos=b,          % подписи снизу -- вроде не надо
  inputencoding=koi8-r,
  numbers=left,          % нумерация слева
  numberstyle=\footnotesize,
  showspaces=false,      % показывать пробелы подчеркиваниями -- идиотизм 70-х годов
  showstringspaces=false,
  showtabs=false,        % и табы тоже
  stepnumber=1,
  tabsize=4,              % кому нужны табы по 8 символов?
  frame=single
}

% Стиль для псевдокода: строчки обычно короткие, поэтому размер шрифта побольше
\lstdefinestyle{pseudocode}{
  basicstyle=\small,
  keywordstyle=\color{black}\bfseries\underbar,
  language=Pseudocode,
  numberstyle=\footnotesize,
  commentstyle=\footnotesize\it
}

% Стиль для обычного кода: маленький шрифт
\lstdefinestyle{realcode}{
  basicstyle=\scriptsize,
  numberstyle=\footnotesize
}

% Стиль для коротких кусков обычного кода: средний шрифт
\lstdefinestyle{simplecode}{
  basicstyle=\footnotesize,
  numberstyle=\footnotesize
}

% Стиль для BNF
\lstdefinestyle{grammar}{
  basicstyle=\footnotesize,
  numberstyle=\footnotesize,
  stringstyle=\bfseries\ttfamily,
  language=BNF
}

% Определим свой язык для написания псевдокодов на основе Python
\lstdefinelanguage[]{Pseudocode}[]{Python}{
  morekeywords={each,empty,wait,do},% ключевые слова добавлять сюда
  morecomment=[s]{\{}{\}},% комменты {а-ля Pascal} смотрятся нагляднее
  literate=% а сюда добавлять операторы, которые хотите отображать как мат. символы
    {->}{\ensuremath{$\rightarrow$}~}2%
    {<-}{\ensuremath{$\leftarrow$}~}2%
    {:=}{\ensuremath{$\leftarrow$}~}2%
    {<--}{\ensuremath{$\Longleftarrow$}~}2%
}[keywords,comments]

% Свой язык для задания грамматик в BNF
\lstdefinelanguage[]{BNF}[]{}{
  morekeywords={},
  morecomment=[s]{@}{@},
  morestring=[b]",%
  literate=%
    {->}{\ensuremath{$\rightarrow$}~}2%
    {*}{\ensuremath{$^*$}~}2%
    {+}{\ensuremath{$^+$}~}2%
    {|}{\ensuremath{$|$}~}2%
}[keywords,comments,strings]

% Подписи к листингам на русском языке.
\renewcommand\lstlistingname{Листинг}
\renewcommand\lstlistlistingname{Листинги}

\else
\usepackage{local-minted}
\fi

% Полезные макросы листингов.
% Любимые команды
\newcommand{\Code}[1]{\textbf{#1}}

\newenvironment{signstabular}[1][1]{
    \renewcommand*{\arraystretch}{#1}
    \tabular
    }{
    \endtabular
}


% Стиль титульного листа и заголовки

\geometry{
    a4paper,
    left=30mm,
    right=10mm,
    top=20mm,
    bottom=20mm
}

\begin{document}



\frontmatter % выключает нумерацию ВСЕГО; здесь начинаются ненумерованные главы: реферат, введение, глоссарий, сокращения и прочее.

%\maketitle %создает титульную страницу
%\begin{titlepage}
    \thispagestyle{empty}

    \noindent\begin{minipage}{0.05\textwidth}
                 \includesvg[width=60pt]{img/bmstu.svg}
    \end{minipage}
    \hfill
    \begin{minipage}{0.85\textwidth}\raggedleft
        \begin{center}
            \fontsize{12pt}{0.3\baselineskip}\selectfont \textbf{Министерство науки и высшего образования Российской Федерации \\ Федеральное государственное бюджетное образовательное учреждение \\ высшего образования \\ <<Московский государственный технический университет \\ имени Н.Э. Баумана \\ (национальный исследовательский университет)>> \\ (МГТУ им. Н.Э. Баумана)}
        \end{center}
    \end{minipage}

    \begin{center}
        \fontsize{12pt}{0.1\baselineskip}\selectfont
        \noindent\makebox[\linewidth]{\rule{\textwidth}{4pt}} \makebox[\linewidth]{\rule{\textwidth}{1pt}}
    \end{center}

    \begin{flushleft}
        \fontsize{12pt}{0.8\baselineskip}\selectfont

        ФАКУЛЬТЕТ {\expandafter \uline{
            Информатика и системы управления
            \hfill}}

        КАФЕДРА \uline{\mbox{\hspace{4mm}}
            Программное обеспечение ЭВМ и информационные технологии
            \hfill}
    \end{flushleft}

    \vfill

    \begin{center}
        \fontsize{20pt}{\baselineskip}\selectfont

        \textbf{ОТЧЕТ}

        \textbf{\textit{К ДОМАШНЕЙ РАБОТЕ}}

        \textbf{\textit{ПО КУРСУ:}}
    \end{center}

    \begin{center}
        \fontsize{18pt}{0.6cm}\selectfont

        Программирование параллельных процессов

    \end{center}

    \vfill
    \vfill

    \begin{table}[h!]
        \fontsize{12pt}{0.7\baselineskip}\selectfont
        \centering
        \begin{signstabular}[0.7]{p{7.25cm} >{\centering\arraybackslash}p{4cm} >{\centering\arraybackslash}p{4cm}}
            Студент группы ИУ7-32М & \uline{\mbox{\hspace*{4cm}}} & \uline{\hfill В.Н. Ларин  \hfill} \\
            & \scriptsize (Подпись, дата) & \scriptsize (И.О. Фамилия)
        \end{signstabular}

        \vspace{\baselineskip}

        \begin{signstabular}[0.7]{p{7.25cm} >{\centering\arraybackslash}p{4cm} >{\centering\arraybackslash}p{4cm}}
            Руководитель & \uline{\mbox{\hspace*{4cm}}} & \uline{\hfill А.П. Ковтушенко \hfill} \\
            & \scriptsize (Подпись, дата) & \scriptsize (И.О. Фамилия)
        \end{signstabular}

        \vspace{\baselineskip}

%        \begin{signstabular}[0.7]{p{7.25cm} >{\centering\arraybackslash}p{4cm} >{\centering\arraybackslash}p{4cm}}
%            Консультант & \uline{\mbox{\hspace*{4cm}}} & \uline{\hfill  \hfill} \\
%            & \scriptsize (Подпись, дата) & \scriptsize (И.О. Фамилия)
%        \end{signstabular}
    \end{table}

%    \vfill

    \begin{center}
        \normalsize \textit{\textbf{2024} г.}
    \end{center}
%\end{titlepage}
\include{01-statement}


%\begin{executors}
%\personalSignature{Первый исполнитель}{ФИО}
%
%\personalSignature{Второй исполнитель}{ФИО}
%\end{executors}


%\listoffigures                         % Список рисунков

%\listoftables                          % Список таблиц

%\NormRefs % Нормативные ссылки 
% Команды \breakingbeforechapters и \nonbreakingbeforechapters
% управляют разрывом страницы перед главами.
% По-умолчанию страница разрывается.

% \nobreakingbeforechapters
% \breakingbeforechapters

%\null\newpage

%\setcounter{page}{3}


\tableofcontents

\printnomenclature % Автоматический список сокращений

\onecolumn

\Introduction


Целью данной работы является разработка программы для вычисления матрицы, обратной заданной, с использованием метода Р-приведения.

Для достижения цели необходимо выполнить следующие задачи:
\begin{itemize}
    \item обосновановать выбор алгоритма решения задачи;
    \item разработать программу для нахождения обратной матрицы методом Р-приведения:
    \item исследование влияния размерности матрицы и числа процессоров на время выполнения программы.
\end{itemize}


Актуальность исследования обусловлена необходимостью разработки эффективных методов для решения задач линейной алгебры, таких как вычисление обратной матрицы, которые могут применяться в различных областях науки и техники.
Обратные матрицы играют ключевую роль во многих математических моделях, например, в системах управления, обработке сигналов, статистическом анализе данных и других приложениях.

Метод Р-приведения является одним из классических подходов к решению подобных задач, однако его эффективность может значительно варьироваться в зависимости от размера задачи и архитектуры вычислительного оборудования.
Современные высокопроизводительные системы требуют оптимизированных решений, способных равномерно распределять нагрузку между несколькими процессорами, что позволяет существенно сократить время расчётов.

Исследование зависимости времени счёта от размерности задачи и количества процессоров позволит выявить закономерности и определить наиболее эффективные параметры для реализации программ на современных многоядерных системах.
Это знание будет полезно не только для улучшения производительности существующих приложений, но и для проектирования новых систем, ориентированных на параллельную обработку данных.




\mainmatter % это включает нумерацию глав и секций в документе ниже

\chapter{Аналитический раздел}


\section{Обратная матрица}

Обратная ма́трица — такая матрица $A^{-1}$, при умножении которой на исходную матрицу $A$ получается единичная матрица $E$:

\begin{equation}
    \label{eq:eq_a1}
    AA^{-1}=A^{-1}A=E.
\end{equation}

Обратную матрицу можно определить как:

\begin{equation}
    \label{eq:eq_a2}
    A^{-1}={\frac {{\mbox{adj}}A}{|A|}},
\end{equation}
где ${\mbox{adj}}A$ — соответствующая присоединённая матрица, $|A|$ — определитель матрицы $A$.

Из этого определения следует критерий обратимости: матрица обратима тогда и только тогда, когда она невырождена, то есть её определитель не равен нулю.
Для неквадратных матриц и вырожденных матриц обратных матриц не существует.


\section{Метод Жордана—Гаусса}


Возьмём две матрицы: саму $A$ и единичную матрицу $E$. Приведём матрицу $A$ к единичной методом Гаусса—Жордана, применяя преобразования по строкам (можно также применять преобразования и по столбцам).
После применения каждой операции к первой матрице применим ту же операцию ко второй.
Когда приведение первой матрицы к единичному виду будет завершено, вторая матрица окажется равной $A^{-1}$.

При использовании метода Гаусса первая матрица будет умножаться слева на одну из элементарных матриц $\Lambda_{i}$ (трансвекцию или диагональную матрицу с единицами на главной диагонали, кроме одной позиции):

\begin{equation}
    \label{eq:eq_a10}
    \Lambda _{1}\cdot \dots \cdot \Lambda _{n}\cdot A=\Lambda A=E\Rightarrow \Lambda =A^{-1}.
\end{equation}

\begin{equation}
    \label{eq:eq_a11}
    \Lambda _{m}={\begin{bmatrix}
                      1& \dots & 0& -a_{1m}/a_{mm}& 0& \dots & 0\\& & & \dots & & & \\0& \dots & 1& -a_{m-1m}/a_{mm}& 0& \dots & 0\\0& \dots & 0& 1/a_{mm}& 0& \dots & 0\\0& \dots & 0& -a_{m+1m}/a_{mm}& 1& \dots & 0\\& & & \dots & & & \\0& \dots & 0& -a_{nm}/a_{mm}& 0& \dots & 1
    \end{bmatrix}}.
\end{equation}

Вторая матрица после применения всех операций станет равна $\Lambda$, то есть будет искомой.



\subsection*{Описание алгоритма}

Алгоритм решения подразделяется на два этапа.


На первом этапе осуществляется прямой ход, в ходе которого происходит следующее:


\begin{enumerate}
    \item Определяется, является ли система уравнений совместной. Для этого среди элементов первого столбца матрицы выбирается ненулевой элемент.
    \item Строка, содержащая этот ненулевой элемент, перемещается в крайнее верхнее положение, становясь первой строкой матрицы.
    \item Ненулевые элементы первого столбца всех нижележащих строк обнуляются путём вычитания из каждой строки первой строки, умноженной на отношение первого элемента этой строки к первому элементу первой строки.
\end{enumerate}

После выполнения указанных преобразований первая строка и первый столбец мысленно вычёркиваются, и процесс повторяется до тех пор, пока не останется матрица нулевого размера.


На втором этапе осуществляется обратный ход, в ходе которого происходит следующее:


\begin{enumerate}
    \item Выражается каждая базисная переменная через небазисные переменные.
    \item Строится фундаментальная система решений.
    \item Если все переменные являются базисными, то выражается единственное решение системы линейных уравнений.
\end{enumerate}

Процедура начинается с последнего уравнения, из которого выражается соответствующая базисная переменная (она там всего одна) и подставляется в предыдущие уравнения.
Затем процесс повторяется для каждого уравнения, поднимаясь по «ступеням» наверх.
Каждой строке соответствует ровно одна базисная переменная, поэтому на каждом шаге, кроме последнего, ситуация повторяется.

Сложность алгоритма -- $O(n^{3})$.



\chapter{Конструкторский раздел}
\label{ch:design}


\section{Проектирование последовательного алгоритма}

Был разработан последовательный алгоритм поиска обратной матрицы, представленный на алгоритме~\ref{alg:inv_sq_matrix}.

\begin{small}
    \begin{algorithm}[H]
        \caption{Последовательный алгоритм поиска обратной матрицы}
        \label{alg:inv_sq_matrix}
        \begin{algorithmic}[1]
            \Procedure{invert\_sq\_matrix}{$matrixsize$, $matrix$, $inv$}
                \State Занулить матрицу $inv\_matrix$

                \For{$i$ от $0$ до $matrixsize-1$}
                    \State $inv[i][i] \gets 1$
                \EndFor

                \For{$i$  от $0$ до $matrixsize-1$}
                    \State $pivot \gets matrix[i][i]$

                    \State Умножить строку $matrix[i]$ на $1/pivot)$
                    \State Умножить строку $inv[i]$ на $1/pivot)$

                    \For{$j$ от $i+1$ до $matrixsize-1$}
                        \State $mul \gets matrix[j][i]$

                        \State Добавить к $matrix[j]$ строку $matrix[i]$, умноженную на $-mul$
                        \State Добавить к $inv[j]$ строку $inv[i]$, умноженную на $-mul$
                    \EndFor
                \EndFor

                \For{$i$ от $matrixsize-1$ до $1$ с шагом $-1$}
                    \For{$j$ от $0$ до $i-1$}
                        \State $mul \gets matrix[j][i]$
                        \State Добавить к $matrix[j]$ строку $matrix[i]$, умноженную на $-mul$
                        \State Добавить к $inv[j]$ строку $inv[i]$, умноженную на $-mul$
                    \EndFor
                \EndFor
            \EndProcedure
        \end{algorithmic}
    \end{algorithm}
\end{small}


\section{Проектирование параллельного алгоритма}

\subsection*{Анализ трудностей разработки}

Параллельный алгоритм для вычисления обратной матрицы методом Гаусса представляет собой интересный и важный пример применения распределённых вычислительных систем.
Его реализация может столкнуться с рядом проблем при комуникации между вычислительными узлами.
Рассмотрим основные сложности.

Одним из ключевых моментов является разбиение исходной матрицы между процессами.
Матрица должна быть разделена таким образом, чтобы каждый процесс получил свою часть работы, но при этом все процессы могли взаимодействовать друг с другом.
Если не учитывать особенности структуры матрицы и характер операций, то можно получить неравномерное распределение нагрузки между процессорами, что приведёт к увеличению временных затрат.

Метод Гаусса требует выполнения ряда последовательных шагов, которые включают обмен данными между процессорами.
Это приводит к необходимости частых коммуникационных операций, таких как передача строк или столбцов матрицы от одного процесса другому.
Частые коммуникации существенно замедляют выполнение программы, так как время передачи данных через сеть может оказаться значительно больше времени обработки самих данных.

Для корректного выполнения метода необходимо синхронизировать работу всех процессов после каждого шага алгоритма.
Это связано с тем, что результаты одной итерации используются в следующей.


\subsection*{Распределение данных между узлами}

Предлагается циклическая рассылка строк матрицы вычислительным узлам.


Существует несколько причин, почему строки матрицы следует отправлять циклически разным вычислительным узлам:


Предположим, у нас есть матрица размером NxN и P вычислительных узлов.
Мы можем разделить матрицу на P частей и отправить каждую часть соответствующему узлу. Например:


\begin{itemize}
    \item Узел 0 получит строки с номерами $0$, $P$, $2P$, ...
    \item Узел 1 получит строки с номерами 1, $P+1$, $2P+1$, ...
    \item И так далее до узла $P-1$, который получит строки $P-1$, $2P-1$, ...
\end{itemize}

Это гарантирует, что каждая строка будет обработана ровно одним узлом, и каждый узел получит равное количество строк для обработки.


Цикличная рассылка строк матрицы разным вычислительным узлам является эффективным способом обеспечения равномерной загрузки узлов, минимизации коммуникаций.
Данный подход широко используется в различных параллельных алгоритмах.


\subsection*{Алгоритм}

Был разработан распределенный алгоритм поиска обратной матрицы c учетом описанных выше замечаний.
Он представлен на алгоритме~\ref{alg:inv_pl_matrix}.


\begin{small}
    \begin{algorithm}[H]
        \caption{Распределенный алгоритм поиска обратной матрицы на нескольких узлах}
        \label{alg:inv_pl_matrix}
        \begin{algorithmic}[1]
            \Procedure{invert\_matrix}{$matrix$, $invert\_matrix$, $matrix\_length$, $rank$, $size$, $matrix\_rows$}

                \State Распределить матрицу $matrix$ по узлам в $matrix\_r$
                \State Создать нулевую матрицу $inv\_matrix$ размером $matrix\_length \times matrix\_rows$

                \For {$i \gets 0$ to $matrix\_rows - 1$}
                    \State $(\&inv\_matrix[i][i \times size + rank] \gets 1$
                \EndFor


                \For {$i$ от $0$ до $matrix\_length - 1$}
                    \Comment Прямой ход

                    \If {$i \% size = rank$}
                        \Comment Проверка принадлежности узлу

                        \State $pivot \gets row[i]$
                        \State Умножить строку $matrix\_r[i / size]$ на $1/pivot$
                        \State Умножить строку $inv\_matrix[i / size]$ на $1/pivot$

                        \State Разослать всем строки $matrix\_r[i / size]$ , $inv\_matrix[i / size]$
                    \EndIf

                    \State Получить строки с индексом i в $row, irow$

                    \For{$j$ от $0$ до $i / size$}
                        \State $mul \gets row[i]$

                        \State Добавить к $matrix\_r[j]$ строку $row$, умноженную на $-mul$
                        \State Добавить к $inv\_matrix[j]$ строку $irow$, умноженную на $-mul$
                    \EndFor
                \EndFor

                \For{$i$ от $matrixsize-1$ до $1$ с шагом $-1$}
                    \Comment Обратный ход

                    \If {$i \% size = rank$}
                        \Comment Проверка принадлежности узлу
                        \State Разослать всем строки $matrix\_r[i / size]$ , $inv\_matrix[i / size]$
                    \EndIf

                    \State Получить строки с индексом $i$ в $row, irow$


                    \For{$j$ от $0$ до $i / size$}
                        \State $mul \gets row[i]$

                        \State Добавить к $matrix\_r[j]$ строку $row$, умноженную на $-mul$
                        \State Добавить к $inv\_matrix[j]$ строку $irow$, умноженную на $-mul$
                    \EndFor
                \EndFor


                \State Собрать с узлов $inv\_matrix$ в матрицу $invert\_matrix$

            \EndProcedure
        \end{algorithmic}
    \end{algorithm}
\end{small}


\section*{Вывод}

В разделе описан разработанный последовательный алгоритм поиска обратной матрицы, а также распределённый алгоритм поиска обратной матрицы на нескольких узлах. Распределённый алгоритм учитывает сложности разработки параллельного алгоритма и направлен на их преодоление.

Основные сложности разработки параллельного алгоритма включают разбиение исходной матрицы между процессами, метод Гаусса, требующий обмена данными между процессами, и необходимость синхронизации работы всех процессов после каждого шага алгоритма для корректного выполнения метода.

Для распределения данных между узлами предлагается использовать циклическую рассылку строк матрицы вычислительным узлам, чтобы каждая строка была обработана ровно одним узлом и каждый узел получил равное количество строк для обработки.

Распределённый алгоритм поиска обратной матрицы включает распределение матрицы по узлам, создание нулевой матрицы, прямой ход, обратный ход, сборку матрицы с узлов. Такой подход позволяет равномерно загрузить узлы, минимизировать коммуникации и эффективно использовать вычислительные ресурсы.
\chapter{Технологический раздел}
\label{cha:impl}


\section{Выбор средств программной реализации}


Основным средством разработки является язык программирования.
Был выбран язык программирования C.
Для выполенения паралелльныхх вычислений была использована библиотека OpenMPI.


\section{Структуры MPI}


Был создан тип данных, представляющий собой непрерывную последовательность элементов заданного базового типа MPI\_DOUBLE.
Этот тип данных удобен для передачи целых строк матрицы.
Данный тип определен следующим образом.

\begin{small}
    \begin{verbatim}
MPI_Datatype mpi_row;
MPI_Type_contiguous(matrix_length, MPI_DOUBLE, &mpi_row);
    \end{verbatim}
\end{small}

Для цикличной рассылки был использован следующий тип.
Эта структура данных описывает создание типа данных для передачи строк матрицы в MPI.
Сначала создается временный тип данных -- вектор из строк матрицы c пробелом в $size-1$ строк.
Затем данный временный тип преобразуется в окончательный тип: задается выравнивание в памяти, чтобы можно было получить следующие $N$ строк для другого узла.
Тип определен следующим образом.


\begin{small}
    \begin{verbatim}
MPI_Datatype mpi_row_sh;
{
    MPI_Datatype mpi_row_tmp;
    MPI_Type_vector(max_matrix_rows, matrix_length,
        matrix_length * size, MPI_DOUBLE, &mpi_row_tmp);
    MPI_Type_create_resized(mpi_row_tmp, 0,
        sizeof(double) * matrix_length, &mpi_row_sh);
    MPI_Type_free(&mpi_row_tmp);
}
    \end{verbatim}
\end{small}


Отправка и сбор данных происходит с помощью функций.


\begin{small}
    \begin{verbatim}
MPI_Scatter(matrix, 1, mpi_row_sh,
            matrix_r, max_matrix_rows, mpi_row,
            0, MPI_COMM_WORLD);

MPI_Gather(inv_matrix, max_matrix_rows, mpi_row,
           invert_matrix, 1, mpi_row_sh,
           0, MPI_COMM_WORLD);
    \end{verbatim}
\end{small}


\section{Организация проверки валидности данных}


Для проверки ошибки вычислений используется следующая функция.


\begin{small}
    \begin{verbatim}
double inv_check(int matrix_len, double *left, double *right) {
    double *tmp = malloc(sizeof(*tmp) * matrix_len * matrix_len);

    mul_matrix(left, right, tmp, matrix_len);

    double diff = 0;
    for (size_t row = 0; row < matrix_len; row++) {
        for (size_t col = 0; col < matrix_len; col++) {
            double data = tmp[row * matrix_len + col];
            if (row == col) diff += fabs(1 - data);
            else diff += fabs(data);
        }
    }

    free(tmp);
    return diff / (matrix_len * matrix_len);
}
    \end{verbatim}
\end{small}


\section*{Вывод}

В данном разделе описывается выбор средств программной реализации для выполнения определённых задач.
Для разработки используется язык программирования C. Для параллельных вычислений применяется OpenMPI.
Для рассылки строк и цкиличной рассылки строк матрицы были разработаны типы данных.

Для контроля ошибок вычислений применяется следующая функция.
Функция inv\_check умножает матрицы, вычисляет разницу между резудьтатом и единичной матрицей.


\chapter{Экспериментальный раздел}
\label{cha:research}


\section{Технические характеристики}
Вычисления проводились на кластере, состоящем из 10 узлов, со следующими техническими характеристиками:
\begin{itemize}
    \item Операционная система Ubuntu 18.04.6 LTS (GNU/Linux 4.15.0 x86\_64)
    \item Оперативная память 4 ГБ
    \item Процессор Intel(R) Xeon(R) Gold 6248 CPU @ 2.50GHz --- 1 ядро
\end{itemize}

Для управления задачами использовались системы управления заданиями Slurm.


\section{Время выполнения алгоритмов}

Для замеров времени использовалась функция MPI \Verb{MPI_Wtime}.

Для одного узла используется последовательнный алгоритм.

Было вычислено ускорение вычислений $S$ по формуле
\begin{equation}
    \label{eq:eq_s}
    S(Size, n) = \frac{Time(Size, n)} {Time(Size, 1)}
\end{equation}
и коэффициент полезной нагрузки $E$ по формуле
\begin{equation}
    \label{eq:eq_e}
    E(Size, n) = \frac{S(Size, n)} {n} ,
\end{equation}
где $Size$ -- размерность квадратной матрицы, $n$ -- количество вычислительных узлов.

Результаты замеров и вычислений представлены в таблице~\ref{tab:results-all-m}.


\begin{table}[H]
    \center

    \caption{Результаты вычислений}
    \label{tab:results-all-m}
    \fontsize{11pt}{11pt}\selectfont
    \begin{tabular}{|l|l|rrrrrrrrrr|}
        \hline
        & PC   & 1      & 2      & 3      & 4      & 5      & 6      & 7      & 8     & 9     & 10    \\
        Size                     &      &        &        &        &        &        &        &        &       &       &       \\
        \hline
        \multirow[t]{3}{*}{100}  & E    & 1.000  & 0.138  & 0.085  & 0.050  & 0.038  & 0.013  & 0.011  & 0.018 & 0.008 & 0.015 \\
        & S    & 1.000  & 0.276  & 0.256  & 0.202  & 0.190  & 0.079  & 0.076  & 0.142 & 0.073 & 0.150 \\
        & Time & 0.006  & 0.023  & 0.025  & 0.032  & 0.034  & 0.081  & 0.085  & 0.045 & 0.088 & 0.043 \\
        \cline{1-12}
        \multirow[t]{3}{*}{200}  & E    & 1.000  & 0.359  & 0.255  & 0.154  & 0.153  & 0.107  & 0.087  & 0.061 & 0.069 & 0.053 \\
        & S    & 1.000  & 0.718  & 0.764  & 0.615  & 0.764  & 0.643  & 0.608  & 0.484 & 0.619 & 0.526 \\
        & Time & 0.049  & 0.068  & 0.064  & 0.080  & 0.064  & 0.076  & 0.081  & 0.101 & 0.079 & 0.093 \\
        \cline{1-12}
        \multirow[t]{3}{*}{400}  & E    & 1.000  & 0.643  & 0.534  & 0.381  & 0.381  & 0.289  & 0.242  & 0.197 & 0.195 & 0.161 \\
        & S    & 1.000  & 1.285  & 1.601  & 1.525  & 1.903  & 1.736  & 1.697  & 1.573 & 1.757 & 1.609 \\
        & Time & 0.381  & 0.296  & 0.238  & 0.250  & 0.200  & 0.219  & 0.224  & 0.242 & 0.217 & 0.237 \\
        \cline{1-12}
        \multirow[t]{3}{*}{600}  & E    & 1.000  & 0.793  & 0.677  & 0.474  & 0.460  & 0.449  & 0.363  & 0.291 & 0.274 & 0.248 \\
        & S    & 1.000  & 1.587  & 2.032  & 1.895  & 2.302  & 2.696  & 2.542  & 2.325 & 2.469 & 2.475 \\
        & Time & 1.283  & 0.809  & 0.631  & 0.677  & 0.557  & 0.476  & 0.505  & 0.552 & 0.520 & 0.518 \\
        \cline{1-12}
        \multirow[t]{3}{*}{800}  & E    & 1.000  & 0.535  & 0.729  & 0.335  & 0.268  & 0.256  & 0.409  & 0.357 & 0.149 & 0.143 \\
        & S    & 1.000  & 1.069  & 2.186  & 1.340  & 1.341  & 1.537  & 2.862  & 2.854 & 1.345 & 1.434 \\
        & Time & 3.015  & 2.819  & 1.379  & 2.249  & 2.248  & 1.962  & 1.053  & 1.056 & 2.241 & 2.102 \\
        \cline{1-12}
        \multirow[t]{3}{*}{1000} & E    & 1.000  & 0.876  & 0.826  & 0.470  & 0.407  & 0.357  & 0.310  & 0.347 & 0.266 & 0.237 \\
        & S    & 1.000  & 1.753  & 2.479  & 1.881  & 2.037  & 2.141  & 2.171  & 2.775 & 2.390 & 2.375 \\
        & Time & 5.881  & 3.356  & 2.372  & 3.127  & 2.888  & 2.747  & 2.709  & 2.119 & 2.461 & 2.477 \\
        \cline{1-12}
        \multirow[t]{3}{*}{1200} & E    & 1.000  & 0.930  & 0.890  & 0.637  & 0.513  & 0.473  & 0.437  & 0.483 & 0.351 & 0.332 \\
        & S    & 1.000  & 1.861  & 2.671  & 2.547  & 2.565  & 2.835  & 3.057  & 3.867 & 3.155 & 3.319 \\
        & Time & 10.517 & 5.652  & 3.937  & 4.129  & 4.100  & 3.709  & 3.441  & 2.720 & 3.334 & 3.169 \\
        \cline{1-12}
        \multirow[t]{3}{*}{1400} & E    & 1.000  & 0.921  & 0.909  & 0.823  & 0.637  & 0.554  & 0.514  & 0.575 & 0.407 & 0.393 \\
        & S    & 1.000  & 1.843  & 2.727  & 3.290  & 3.183  & 3.324  & 3.595  & 4.596 & 3.667 & 3.931 \\
        & Time & 16.734 & 9.081  & 6.137  & 5.086  & 5.257  & 5.034  & 4.655  & 3.641 & 4.563 & 4.257 \\
        \cline{1-12}
        \multirow[t]{3}{*}{1600} & E    & 1.000  & 0.937  & 0.920  & 0.861  & 0.700  & 0.663  & 0.596  & 0.601 & 0.535 & 0.487 \\
        & S    & 1.000  & 1.873  & 2.760  & 3.444  & 3.502  & 3.981  & 4.174  & 4.810 & 4.814 & 4.873 \\
        & Time & 25.403 & 13.562 & 9.204  & 7.377  & 7.254  & 6.382  & 6.086  & 5.281 & 5.277 & 5.213 \\
        \cline{1-12}
        \multirow[t]{3}{*}{1800} & E    & 1.000  & 0.922  & 0.901  & 0.792  & 0.715  & 0.730  & 0.625  & 0.591 & 0.545 & 0.528 \\
        & S    & 1.000  & 1.844  & 2.704  & 3.170  & 3.576  & 4.379  & 4.378  & 4.726 & 4.905 & 5.279 \\
        & Time & 35.814 & 19.422 & 13.244 & 11.299 & 10.014 & 8.179  & 8.181  & 7.578 & 7.301 & 6.784 \\
        \cline{1-12}
        \multirow[t]{3}{*}{2000} & E    & 1.000  & 0.929  & 0.910  & 0.855  & 0.789  & 0.689  & 0.688  & 0.650 & 0.602 & 0.599 \\
        & S    & 1.000  & 1.857  & 2.729  & 3.419  & 3.946  & 4.133  & 4.813  & 5.202 & 5.417 & 5.988 \\
        & Time & 49.325 & 26.559 & 18.077 & 14.425 & 12.501 & 11.935 & 10.248 & 9.481 & 9.106 & 8.238 \\
        \cline{1-12}
        \hline
    \end{tabular}
\end{table}


На рисунке~\ref{fig:graph} показан график зависимости времени работы метода от размерности задачи для разного количества вычислительных узлов.

\begin{figure}[H]
    \centering
    \includesvg[width=\textwidth]{img/graph.svg}
    \caption{Зависимость времени работы метода от размерности задачи для разного количества вычислительных узлов}
    \label{fig:graph}
\end{figure}

Согласно полученным результатам можно сделать вывод о том, что использование более одного вычислительного узла оправдано при вычислениях с матрицами размерностью более 200x200.


Были построены тепловые карты зависимости количества вычислительных узлов от размерности задачи для ускорения вычислений $S$ и коэфициента полезной нагрузки $E$.
Данные карты предствлены на рисунке~\ref{fig:heatmap_s}.

\begin{figure}[H]
    \centering
    \begin{minipage}[h]{0.49\textwidth}
        \center{\includesvg[width=1\textwidth]{img/heatmap_s.svg} \\ a)}
    \end{minipage}
    \hfill
    \begin{minipage}[h]{0.49\textwidth}
        \center{\includesvg[width=1\textwidth]{img/heatmap_e.svg} \\ б)}
    \end{minipage}
    \caption{Тепловая карта зависимости количества вычислительных узлов от размерности задачи для \\ a) ускорения вычислений  $S$  б) коэфициента полезной нагрузки  $E$   }
    \label{fig:heatmap_s}
\end{figure}


Из данных тепловых карт можно сделать вывод о том, что до 40\% времени работы алгоритма уходит на операции распределения данных между вычислительными узлами и процесссов их синхронизации.


\section*{Вывод}

Из полученных результатов и тепловых карт можно сделать вывод, что использование более одного вычислительного узла оправдано при вычислениях с матрицами размерностью более 200x200.
До 40\% времени работы алгоритма уходит на операции распределения данных между вычислительными узлами и процессов их синхронизации.




\backmatter %% Здесь заканчивается нумерованная часть документа и начинаются ссылки и
            
\Conclusion % заключение к отчёту


Для достижения цели были выполнены следующие задачи.


\begin{enumerate}
\item Обоснован выбор алгоритма решения задачи.
\item Разработана программа для нахождения обратной матрицы методом Р-приведения.
\item Проведено исследование влияния размерности матрицы и числа процессоров на время выполнения программы.
\end{enumerate}



%В ходе работы получены следующие результаты.
%Была раскрыта сущность проблемы композиции программных интерфейсов.
%Выделены необходимые для композиции способы преобразования данных в виде операций над сущностями схемы API.
%Проведен обзор протоколов передачи данных и приведены их особенности при выполнении преобразований данных.
%
%Была выполнена классификация методов композиции программных интерфейсов в сервис-ориентированной архитектуре.
%Проведен сравнительный анализ методов и на его основании было выявлено, что полную поддержку операций над сущностями предоставляют методы композиции с помощью графа сущностей, агрегирующего хранилища и на уровне клиентского приложения.
%Методы агрегирующего хранилища и на уровне клиента не могут динамически перестраивать схему композиции.
%
%Был выполнен обзор существующих программных продуктов и на основании их анализа выявлены недостатки.
%Большинство продуктов предоставляются как облачное решение, что сужает круг применения.
%Решения Apollo Federation и Anypoint Platform предлагает полную поддержку операций над сущностями в виду поддержки композиции с помощью графа сущностей.
%
%Была проведена формализация задачи композиции программных интерфейсов в нотации IDEF0.
%
%Результаты работы показывают необходимость разработки решений для устранения следующих проблем.
%\begin{itemize}
%    \item Существует малое количество коробочных решений для композиции API с полной поддержкой операций над сущностями для протокола REST, который является одним из основных для построения веб-приложений.
%    \item Существующие решения композиции для протокола GraphQL наиболее обширны в покрытии выделенных операций над сущностями, но не обладают эффективными схемами построения множественных запросов.
%\end{itemize}
%
%Все поставленные задачи были решены.
%Таким образом цель данной работы была достигнута.
%Результаты работы могут найти применение в разработке метода композиции программных интерфейсов.

%%% Local Variables: 
%%% mode: latex
%%% TeX-master: "rpz"
%%% End: 
%% заключение


% % Список литературы при помощи BibTeX
% Юзать так:
%
% pdflatex rpz
% bibtex rpz
% pdflatex rpz

\bibliographystyle{ugost2008}
\bibliography{rpz}

%\printbibliography

%%% Local Variables: 
%%% mode: latex
%%% TeX-master: "rpz"
%%% End: 



%\appendix   % Тут идут приложения

%\chapter{Презентация}
\label{cha:appendix1}

%\blindtext
%\begin{figure}
%\centering
%\caption{Картинка в приложении. Страшная и ужасная.}
%\end{figure}

%%% Local Variables: 
%%% mode: latex
%%% TeX-master: "rpz"
%%% End: 

%
%\chapter{Еще картинки}
\label{cha:appendix2}
\blindtext

\begin{figure}
\centering
\caption{Еще одна картинка, ничем не лучше предыдущей. Но надо же как-то заполнить место.}
\end{figure}

%%% Local Variables: 
%%% mode: latex
%%% TeX-master: "rpz"
%%% End: 


\end{document}

%%% Local Variables:
%%% mode: latex
%%% TeX-master: t
%%% End:
